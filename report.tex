\documentclass{article}
\usepackage[utf8]{inputenc}

\usepackage[utf8]{inputenc}
\usepackage[T2B]{fontenc}
\usepackage[english,russian]{babel}
\frenchspacing

\usepackage{amsmath}
\usepackage{amssymb}
\usepackage{hyperref}
\usepackage{longtable}
\usepackage[table]{xcolor}
\usepackage{array}
\usepackage{color}
\usepackage{xcolor}

\usepackage{hyperref}


\newcommand{\MYhref}[3][blue]{\href{#2}{\color{#1}{#3}}}%

\usepackage{listings}
\usepackage{alltt}
\usepackage{csquotes}
\DeclareQuoteStyle{russian}
	{\guillemotleft}{\guillemotright}[0.025em]
	{\quotedblbase}{\textquotedblleft}
\ExecuteQuoteOptions{style=russian}

\usepackage{graphicx}

\usepackage{listings}
\lstset{tabsize=2,
	breaklines,
	columns=fullflexible,
	flexiblecolumns,
	numbers=left,
	numberstyle={\footnotesize},
	extendedchars,
	inputencoding=utf8}

\usepackage{longtable}

\def\@xobeysp{ }
\def\verbatim@processline{\hspace{1.2cm}\raggedright\the\verbatim@line\par}

\oddsidemargin=-0.4mm
\textwidth=160mm
\topmargin=4.6mm
\textheight=210mm

\parindent=0pt
\parskip=3pt

\definecolor{lightgray}{gray}{0.9}


\renewcommand{\thesubsection}{\arabic{subsection}}

\lstdefinestyle{customc}{
  belowcaptionskip=1\baselineskip,
  breaklines=true,
  frame=L,
  xleftmargin=\parindent,
  language=C,
  showstringspaces=false,
  basicstyle=\footnotesize\ttfamily,
  keywordstyle=\bfseries\color{green!40!black},
  commentstyle=\itshape\color{gray},
  identifierstyle=\color{black},
  stringstyle=\color{blue},
}

\lstdefinestyle{customasm}{
  belowcaptionskip=1\baselineskip,
  frame=L,
  xleftmargin=\parindent,
  language=[x86masm]Assembler,
  basicstyle=\footnotesize\ttfamily,
  commentstyle=\itshape\color{purple!40!black},
}

\lstset{escapechar=@,style=customc}


\newcommand{\CWPHeader}[1]{\addtocounter{section}{-1}\section{#1}}

% Заголовок лабораторной работы.
% Единственный аргумент --- ее тема
\newcommand{\CWHeader}[1]{\section*{#1}}

\newcommand{\CWProblem}[1]{\par\textbf{Задача: }#1}
\begin{document}
\begin{titlepage}
\begin{center}
\bfseries

{\Large Московский авиационный институт\\ (национальный исследовательский университет)

}

\vspace{48pt}

{\large Факультет информационных технологий и прикладной математики
}

\vspace{36pt}

{\large Кафедра вычислительной математики и~программирования

}


\vspace{48pt}

Лабораторная работа \textnumero 1 по курсу \enquote{Дискретный анализ}

\end{center}

\vspace{72pt}

\begin{flushright}
\begin{tabular}{rl}
Студент: & Т.\,А. Бердикин \\
Преподаватель: & А.\,А. Кухтичев \\
Группа: & М8О-207Б \\
Дата: & \\
Оценка: & \\
Подпись: & \\
\end{tabular}
\end{flushright}

\vfill

\begin{center}
\bfseries
Москва, \the\year
\pagebreak
\end{center}
\end{titlepage}
\CWHeader{Лабораторная работа \textnumero 1}

\CWProblem{
Требуется разработать программу, осуществляющую ввод пар \enquote{ключ-значение}, их 
упорядочивание по возрастанию ключа указанным алгоритмом сортировки за линейное время и вывод отсортированной последовательности.

{\bfseries Вариант сортировки:} Сортировка подсчётом.

{\bfseries Вариант ключа:} { \normalfont\ttfamily числа от 0 до $10^{6}-1$, почтовые индексы. }

{\bfseries Вариант значения:} { \normalfont\ttfamily числа от 0 до $2^{64}-1$.}
}
\pagebreak
\section{Описание}

Основная идея сортировки подсчётом, алгоритма со списком. Этот вариант используется, когда на вход подается массив структур данных, который следует отсортировать по ключам (назовём их key, а максимальный ключ пусть будет k). Нужно создать вспомогательный массив $C[0..k - 1]$, каждый $C[i]$ в дальнейшем будет содержать список элементов из входного массива. Затем последовательно прочитать элементы входного массива $A$, каждый $A[i]$ добавить в список $C[A[i].key]$. В заключении пройти по массиву $C$, для каждого $j \in [0..k - 1]$ в массив A последовательно записывать элементы списка $C[j]$. Алгоритм устойчив - это значит, что он не будет менять относительный порядок сортируемых элементов, имеющих одинаковые ключи. \site{Wiki}.

\pagebreak

\section{Исходный код}
Входные данные состоят из пар: ключ, значение, где ключ - это почтовый индекс, а значение - это 64-битное число в десятичном представлении. Так как длинна нашего ключа ничем не зафиксирована, то необходимо его привести к какому-то единому стандарту. Этим стандартом является число длиной в 6 символов,так как это максимальная длинна почтового индекса. Значит, что? Если во входных данных индекс имеет менее шести знаков, чтобы было красиво, допишем незначащие нули спереди c помощью функций $setw$ и $setfill$ из библиотеки $<iomanip>$. Создадим класс $TInsex$ в котором будут храниться ключи и значения, функция $Set$, которая поможет нам в создании пар ключ-значение, функция $Show$, выводящая элементы массива. Также дружественной функцией туда записана сортировка подсчётом.
Теперь самое интересное - функция $main$. Узнал, что существенно может ускорить программу рассинхронизация потоков с помощью строки $ios::sync_with_stdio(false);$. Дальше всё просто: создаём переменные ключа и значения, потом вектор с лаконичным названием $a$, а затем переменную tmp, которая и будет парой, объединяющей ключ и значение. Вводим ключи и значения через, неожиданно, $scanf$, потому что это быстрее. Неоднократно проверено с помощью $chrono$, о котором подробнее в разделе \enquote{Тест производительности}. Скорее всего, это связано с тем, что cin и cout должны синхронизироваться со стандартной библиотекой Си. Затем применяем функцию $Set$, а затем получившуюся пару помещаем в вектор. Повторяем ввод и нижеописанные действия, пока не введём всё, что нам нужно. Затем всё сортируем и выводим.


\begin{lstlisting}[language=C++]
#include <iostream>
#include <iomanip>
#include <algorithm>
#include <cassert>

using namespace std;

const int SIZE_ARRAY = 1000000;

template <typename T>
class TVector {
public:
    using value_type = T;
    using iterator = value_type*;
    using const_iterator = const value_type*;

    TVector():
        already_used_(0), storage_size_(0), storage_(nullptr)
    {
    }

    TVector(int size, const value_type& default_value = value_type()):
        TVector()
    {
        assert(size >= 0);

        if (size == 0)
            return;

        already_used_ = size;
        storage_size_ = size;
        storage_ = new value_type[size];

        fill(storage_, storage_ + already_used_, default_value);
    }

    int Size() const
    {
        return already_used_;
    }

    /*bool empty() const
    {
        return Size() == 0;
    }

    iterator begin() const
    {
        return storage_;
    }

    iterator end() const
    {
        if (storage_)
            return storage_ + already_used_;

        return nullptr;
    }*/

    friend void Swap(TVector& lhs, TVector& rhs)
    {
        using std::swap;

        swap(lhs.already_used_, rhs.already_used_);
        swap(lhs.storage_size_, rhs.storage_size_);
        swap(lhs.storage_, rhs.storage_);
    }

    TVector& operator=(TVector other)
    {
        Swap(*this, other);
        return *this;
    }

    TVector(const TVector& other):
        TVector()
    {
        TVector next(other.storage_size_);
        next.already_used_ = other.already_used_;

        if (other.storage_ )
            copy(other.storage_, other.storage_ + other.storage_size_,
                      next.storage_);

        Swap(*this, next);
    }

    ~TVector()
    {
        delete[] storage_;

        storage_size_ = 0;
        already_used_ = 0;
        storage_ = nullptr;
    }

    void PushBack(const value_type& value)
    {
        if (already_used_ < storage_size_) {
            storage_[already_used_] = value;
            ++already_used_;
            return;
        }

        int next_size = 1;
        if (storage_size_)
            next_size = storage_size_ * 2;

        TVector next(next_size);
        next.already_used_ = already_used_;

        if (storage_ )
            std::copy(storage_, storage_ + storage_size_, next.storage_);

        next.PushBack(value);
        Swap(*this, next);
    }

    const value_type& At(int TInsex) const
    {
        if (TInsex < 0 || TInsex > already_used_)
            throw std::out_of_range("You are doing this wrong!");

        return storage_[TInsex];
    }

    value_type& At(int TInsex)
    {
        const value_type& elem = const_cast<const TVector*>(this)-> At(TInsex);
        return const_cast<value_type&>(elem);
    }

    const value_type& operator[](int TInsex) const
    {
        return At(TInsex);
    }

    value_type& operator[](int TInsex)
    {
        return At(TInsex);
    }

private:
    int already_used_;
    int storage_size_;
    value_type* storage_;
};

typedef unsigned long long long_t;

class TInsex{
public:
    long_t key, elem;
    TInsex(int k, long_t e) : key(k), elem(e) {}
    TInsex() : key(), elem() {}
    friend TInsex* CountingSort(TInsex* obj, long_t max, size_t n);
    void Show(){
      cout << setw(6) << setfill('0') << key << "    " << elem << '\n';
    }
    TInsex Set(long_t k, long_t e){
      elem = e;
      key  = k;
      return *this;
    }
  };

void CountingSort(TVector <TInsex>& a){     
  int *c = new int[SIZE_ARRAY]{0};   
  TVector <TInsex> out(a.Size());  

  for (size_t i = 0; i < a.Size(); ++i){     
    c[a[i].key]++;
    //c[a[i].key] = a[i].elem;
  }
  for (int i = 1; i < SIZE_ARRAY; ++i){
    c[i] += c[i - 1];
  }
  for (int i = a.Size() - 1; i >= 0; i--){     
    out[--c[a[i].key]] = a[i];
  }
  Swap(out, a);         
  delete[] c;
}

int main(){
    ios::sync_with_stdio(false); 
    long_t e, k;
    TVector <TInsex> a;
    TInsex tmp;   

    while (scanf("%llu%llu", &k, &e) == 2) {
      tmp.Set(k, e);
      a.PushBack(tmp);    
    }

    CountingSort(a); 

    for (int i = 0; i < a.Size(); i++){
      a[i].Show();
    }
    return 0;
}
\end{lstlisting} 
\pagebreak

\section{Консоль}
\begin{alltt}
tim@wrangler:~/Desktop/da/1$ g++ main.cpp 
tim@wrangler:~/Desktop/da/1$ cat test1.txt 
534095 345092598342
533553 539080594648
354253 532555523555
525 5324534
098432 3542098
43444 423904
1 1
525 5487369
tim@wrangler:~/Desktop/da/1\$ ./a.out < test1.txt 
000001    1
000525    5324534
000525    5487369
043444    423904
098432    3542098
354253    532555523555
533553    539080594648
534095    345092598342
\end{alltt}
\pagebreak
\section{Тест производительности}
В тесте производительности пришлось реализовать целых две программы, в каждой из которых генерируются ключи для сортировки, количество которых, в свою очередь, задаётся пользователем. Для замера времени сортировки используются функции из библиотеки $<chrono>$. Они замеряют время до начала сортировки и сразу после конца, вычитают одно из другого и выводят результат в миллисекундах. Забавно, что время считается от 1.01.1970. Сравнивал я, собственно, сортировку подсчётом и $std::sort$, то бишь, сортировку Хоара.

\begin{alltt}
tim@wrangler:~/Desktop/da/1$ ./count
10
The time: 6 ms
tim@wrangler:~/Desktop/da/1$ ./count
100
The time: 7 ms
tim@wrangler:~/Desktop/da/1$ ./count
1000
The time: 8 ms
tim@wrangler:~/Desktop/da/1$ ./count
10000
The time: 12 ms
tim@wrangler:~/Desktop/da/1$ ./count
100000
The time: 28 ms
tim@wrangler:~/Desktop/da/1$ ./count
1000000
The time: 82 ms
tim@wrangler:~/Desktop/da/1$ ./count
10000000
The time: 838 ms
tim@wrangler:~/Desktop/da/1$ ./count
100000000
The time: 10692 ms
tim@wrangler:~/Desktop/da/1$ ./std 
10
The time: 0 ms
tim@wrangler:~/Desktop/da/1$ ./std 
100
The time: 0 ms
tim@wrangler:~/Desktop/da/1$ ./std 
1000
The time: 0 ms
tim@wrangler:~/Desktop/da/1$ ./std 
10000
The time: 10 ms
tim@wrangler:~/Desktop/da/1$ ./std 
100000
The time: 29 ms
tim@wrangler:~/Desktop/da/1$ ./std 
1000000
The time: 334 ms
tim@wrangler:~/Desktop/da/1$ ./std 
10000000
The time: 3828 ms
tim@wrangler:~/Desktop/da/1$ ./std 
100000000
The time: 41909 ms
\end{alltt}

Из результатов могу выделить то, что при этих тестах, работает быстрее сортировка подсчётом, чем Хоара. Видно, что сложность соритровки Хоара $O(nlogn),$ а сортировки подсчётом - $O(n)$. Cответственно, при большом количестве тестов сортировка подсчётом работает быстрее, чем сортировка Хоара, а при малом - наоборот, что видно при тестах.



\pagebreak
\section{Выводы}
Выполнив первую лабораторную работу по курсу \enquote{Дискретный анализ}, я научился работать со строками, выделяя нужное и преобразуя их в однотипные значения, и вставлять их в вектор. Также я реализовал сортировку подсчетом. Кроме того я сравнил её с сортировкой Хоара, вызывая встроенную функцию std::sort.
\pagebreak
\begin{thebibliography}{99}
\bibitem{Kormen}
Томас\,Х.\,Кормен, Чарльз\,И.\,Лейзерсон, Рональд\,Л.\,Ривест, Клиффорд\,Штайн.
{\itshape Алгоритмы: построение и анализ, 2-е издание.} --- Издательский дом \enquote{Вильямс}, 2007. Перевод с английского: И.\,В.\,Красиков, Н.\,А.\,Орехова, В.\,Н.\,Романов. --- 1296 с. (ISBN 5-8459-0857-4 (рус.))
\bibitem{wikipedia_sort}
{\itshape Сортировка подсчётом - Википедия.} \\URL: \texttt{http://ru.wikipedia.org/wiki/Сортировка\_подсчётом}
\end{thebibliography}
\pagebreak
\end{document}
